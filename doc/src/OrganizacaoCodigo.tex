\documentclass[onecolumn,12pt]{article}

%Adiciona suporte a cores
\usepackage{color}
%\usepackage[brazilian]{babel}

%traduz palavras utilizadas por padrão para português
%\usepackage[portuguese]{babel}

%Adiciona suporte a acentuação, na hora de escrever
\usepackage[utf8]{inputenc}

%Adiciona suporte a acentuação na hora de copiar texto
\usepackage[T1]{fontenc}



%\title{Documenta\c{c}\~{a}o - Comandos de Makefile}

\title{Documentação - Organização de Código}

%\date{2018-02-21}
\author{João Vitor Antoniassi Segantin}

\begin{document}
	\maketitle
%	\tableofcontents
	\begin{abstract}
		Este documento visa explicar a organização de códigos fonte, e oferecer um breve resumo sobre a importância de cada arquivo de código-fonte.\par
		Ao final da leitura desse documento, você será capaz de decidir posições novas de códigos-fonte oriundos de atualizações, implementações de funcionalidades, e/ou re-organizações possíveis.
		Seja bem-vindo(a) ao projeto Visu-RA!
	\end{abstract}
	\newpage
	\indent
	
    \section{Da organização}
    	A organização dos códigos é dada por sua utilidade, seu efeito no projeto.\par
    	Cada arquivo foi separado intencionalmente, a fim de facilitar possíveis extensões, e mudanças de arquiteturas, visando um baixo custo e alto desempenho, a manutenção da organização é de vital importância a todo o projeto. \par
		A interface do banco de dados, em sua maioria está implementada em linguagem C de programação, utilizando definições compatíveis com o gcc (GNU C Compiler), com exceção dos arquivos contidos no diretório "scripts/", que foram implementados em linguagens de programação alternativas, visando maior velocidade de desenvolvimento do projeto.\par
    	Faz-se necessário entender que, os códigos de banco de dados, e os códigos de lógica, estão separados intencionalmente, a fim de obter futura integração com um sistema de Big-Data, independende de Banco de Dados.\par
    	O projeto da interface completo, está situado num repositório privado em bitbucket, caso queira ter acesso a atualizações e coisas do tipo, peça acesso ao seu superior, informando o motido do acesso, seus dados pessoais e quando, precisará desse acesso.\par
    	Como todo repositório git, esse projeto está subversionado, com o intuito de evitar problemas com más implementações. Antes de solicitar um "merge" ou dar "commit", assegure-se de que pelo ou menos duas pessoas tenham visto e autorizado seu commit.\par
    	Lembre-se, está mechendo com milhões de dados, e operações importantes.
    
    	\section{Arquivos de Bancos de Dados}
    		Os arquivos com relação a banco de dados são divididos, em duas partes básicas, os \textbf{arquivos SQL}, que servem para a criação do banco de dados em si, alterações diretas em banco e arquivos de inserção de dados fictíceos para testes, e os \textbf{arquivos de Código-Fonte}, arquivos headers, e bibliotecas do projeto para manipulação do banco de dados.\par
    		Os arquivos de código-fonte, do banco de dados, foram separados do projeto inicial, para facilitar a implementação de um futuro Big-Data.
    		\subsection{Arquivos SQL}
    			Os arquivos SQL estão localizados no diretório \textbf{Banco/}, são eles:
    			\begin{itemize}
    				\item Banco-final.sql\par
    					Utilizado para a geração do banco de dados MySQL.
    				\item Banco-raw.sql\par
    					Arquivo oriundo do "POST-FORWARD script" da ferramente "MySQL Workbench", pode também criar o banco, mas, este ficará com nome "mydb", que não é o padrão reconhecido pela interface. Sua utilização para criação de um banco de dados é recomendada, em caso de re-implementação, ou alteração da estrutura do banco, utilizando a ferramente MySQL workbench. Esse deve ser alterado e salvo como Banco-final.sql, substituindo as ocorrências de "mydb" por "teste"
    				\item produzir-dados.sql\par
    					Script para criação dos dados essenciais para a utilização do Banco de dados pela interface. Cria somente os dados necessários.
    				\item produzir-dados-testes.sql\par
    					Script para criação de dados de testes. Para seu funcionamento ideal, é necessário que "produzir-dados.sql" seja chamado anteriormente.
    			\end{itemize}
    		\subsection{Arquivos de Código-Fonte}
    			Os arquivos de código-fonte estão contidos no diretório \textbf{OperacoesBanco/}, são eles:
    			\begin{itemize}
    				\item OperacoesBanco.h\par
    					Arquivo que contém funções específicas para obtenção, adição e alteração de dados ao banco de dados.
    				\item OperacoesBanco-Visualizacao.h\par
    					Arquivo que contém funções específicas para obtenção, adição e alteração de dados ao banco de dados, especificamente relacionados a visualizações.
    				\item OperacoesBanco-FuncoesGenericas.h\par
    					Arquivo que contém funções generalizadas, utilizadas por todo o restante do código, como conectarBanco().
    			\end{itemize}
	
		\section{Arquivos responsáveis pela lógica}
			Os arquivos a seguir, são responsáveis pela lógica da interface. Responsáveis como, por exemplo, checar se os parâmetros de entrada são viáveis, interpretá-los, montar querys para serem executadas em banco de dados. Também cuidam de permissões de usuário, fazem limitações de utilização de recursos simultâneamente, checam conexão com o banco e coisas do tipo.\par
			Mas, como os arquivos de organização de lógica são, de fato, a maioria do código existente, eles precisam também, de ter subdivisões, que serão explicadas nesse documento.\par
			De modo geral, os arquivos para controle(a lógica) estão contidos dentro dos diretórios \textbf{"Comandos/"}, \textbf{"Criptografia/"}, \textbf{"Fila/"} e \textbf{"scripts/"} (esta ultima, merece uma atenção especial) salvo algumas exceções que não se encaixaram nessa definição, que são os arquivos: \textbf{AdaptadorDeString.h}, \textbf{Server.h}, \textbf{Server.c}, \textbf{Usuario.h} e \textbf{Makefile}.\par
			Ainda, no diretório \textbf{Comandos/}, existe mais uma subdivisão, menos aparente. São os arquivos responsáveis pela lógica de Adição de dados, Obtenção de dados, Remoção de dados, Comandos Administrativos, e arquivos de dissipação (aqueles que decide qual a operação deve ser invocada).

			\subsection{Arquivos da Lógica de Usuário}
				O arquivo que cuida da estrutura e funções de Usuario, intuitivamente é o \textbf{Usuario.h}.\par
				Esse arquivo está localizado no diretório raiz do projeto, e possui funções de controle de usuario, tais como checagem de login e senha, verificador de permissões, construtores e destrutores.\par
				É parte essencial do projeto, e está em frequente atualização.

			\subsection{Arquivos de Lógica de Adição/Alteração de dados}
				O arquivo principal de controle de adição e/ou alteração de dados é o arquivo \textbf{"Comandos/Comando-Adicao.h"}.\par
				Esse arquivo cuida da lógica responsável por adicionar quaisquer tipos de dados, como informações a usuário, usuários, visualizações, produtos, empresas...\par
				Esse arquivo também limita permissões antes de qualquer alteração.

			\subsection{Arquivos de Lógica de Obtenção de dados}
				Existem dois arquivos para controle de obtenção dos dados. São eles:\par
				\begin{itemize}
					\item{"Comandos/Comando-Obter.h"}\par
						Cuida da lógica de obtenção de dados gerais, como "obterTop10NovosProdutos()", ou funções para obter informações de produto, id de localização e coisas do tipo.
					\item{"Comandos/Comando-Obter\_Visualizacoes.h"}\par
						Cuida da lógica da obtenção de informações relacionadas a visualizações, como funções para obter a quantidade de visualizações de usuários anônimos ao produto X.
				\end{itemize}

			\subsection{Arquivos de Lógica de Remoção de dados}
				O arquivo responsável pela lógica de remoção de dados, é o \textbf{"Comandos/Comando-Remover.h"}.\par
				Atualmente, nenhuma função foi implementada ainda, mas em breve, será o arquivo que conterá funções para remoção de dados do banco de dados. Com o exemplo de uma função para remover o registro de um cliente do banco de dados, ou remover um produto da lista de produtos.

			\subsection{Arquivos de Lógica de comandos Administrativos}
				Esse arquivo é um pouco delicado. Segurança deve vir em primeiro lugar, pois é o comando responsável pela execução de comandos que podem ser executados diretamente no SHELL do computador que estiver executando a interface, de fato. Lembre-se que, a interface está sendo executada em modo núcleo, portanto, tenha sensatez ao alterar esse arquivo.\par
				Esse tão temido arquivo é o \textbf{"Comandos/Comando-Root.h"}. Suas funções foram brevemente explicadas acima, para mais informações veja o arquivo de documentação referente ao arquivo.

			\subsection{Arquivos de Lógica de dissipação.}
				Esse arquivo, interpretará a base do comando recebido, e então decidirá se está apto ou não para seguir com sua interpretação.\par
				Também é responsável por checar se o usuário está logado, e algumas outras informações de controle.\par
				Após a identificação de um comando viável, o controle sobre o que fazer com o comando recebido é passado, de acordo com as definições descritas no arquivo de Macros, por esse arquivo, ao arquivo correspondente, responsável pela operação identificada.\par
				Esse arquivo é identificado por \textbf{"Comandos/interpretadorDeComandos.h"}.

			\subsection{Arquivos Principais (onde tudo acontece)}
				Esses arquivos são de grande importância, e são brevemente explicados a seguir:
				\begin{itemize}
					\item{Server.c}\par
						Arquivo que contém a função "main()" do projeto, e controla a ordem de inicialização da interface, e despacha threads a cada nova conexão identificada.

					\item{Server.h}\par
						Arquivo que contém a lógica de interpretação principal, e a função "Servidor", responsável por toda a interpretação, leitura de dados do cliente, e retorno ao mesmo.\par
						Também contém algumas poucas macros, necessárias para a organização de algumas variáveis específicas. É nesse arquivo que a estrutura de Usuario é de fato, armazenada, em tempo de execução. Esse código chama o \textbf{"Comandos/interpretadorDeComandos.h"} caso julgue necessário.

					\item{Makefile}\par
						Arquivo responsável pela lógica dos arquivos de compilação, ele contém os scripts de ordenação de arquivos e dependências de código que são resolvidas em tempo de compilação.\par
						Esse arquivo é responsável pela compilação do projeto, de modo geral, incluindo sua documentação.

					\item{AdaptadorDeString.h}\par
						Esse arquivo cuida de adaptar strings que venham mal configuradas, com acentuação e coisas do tipo, a fim de aumentar a segurança do projeto. Ele é usado em quase todos os códigos do projeto.\par
						Também é responsável pela conversão de inteiros para strings, comparações otimizadas de tamanho de strings e detecção de mensagens de escape.
						Esse arquivo também inclui o "Fila/Fila.h", necessário para uma adaptação de strings de forma otimizada e simplificada utilizando filas.

				\end{itemize}

		\section{Arquivos de Criptografia}
			Esses arquivos, são responsáveis pela criptografia de mensagens recebidas e/ou enviadas, utilizando o algorítimo RSA para isso.\par
			Também é responsável pela criação de novas chaves RSA para cada usuário, e aceitação de conexão segura.
			Estão localizados em \textbf{"Criptografia/"}.
			O arquivo principal é \textbf{"Criptografia/Criptografia.h"}

		\section{Arquivos responsáveis por Macros}
			Como, a maioria das macros utilizadas pelo projeto estão contidas dentro dos arquivos de Comandos, foi então gerado um arquivo de definição de macros de compilação. Utilizadas a fim de diminuir a quantidade de memoria utilizada em tempo de execução, e otimizar algumas operações.\par
			O arquivo principal para definições de Macros:
			\begin{itemize}
				\item{"Comandos/Comandos.h"}\par
					Esse arquivo, futuramente será separado em alguns outros arquivos de definição de macros de ambiente, de erro, de retorno, e de controle.\par
			\end{itemize}
			Esse arquivo contém a maioria das macros necessárias, como string para conexão com banco de dados, números de códigos definidos, mapeamento de erros e de retornos, e coisas do tipo.

		\section{Arquivos de Script}
			São os arquivos localizados no diretório \textbf{"scripts/"}, eles contém funções que podem ser executadas num processo distinto, como ao exemplo, confirmar e-mails recebidos como login, ou monitorar o banco de dados. Os arquivos atualmente são:
			\begin{itemize}
				\item ConfirmadorDeEmail/confirmador.rb\par
					Script responsável por enviar e-mails e confirmar se eles são, de fato, viáveis e existentes. É invocado pela interface. Escrito em RUBY
				\item DatabaseMonitor/monitor.rb\par
					Script responsável por monitorar o banco de dados, gerando LOGs no mesmo, para obter a "quantidade de visualizações naquele momento", é chamado assim que a interface é iniciada. Escrito em RUBY.
				\item SimuladorDeCriacaoDeProdutos/simulador.rb\par
					Script que simula uma série de produtos que seriam criados em tempos diferentes, diretamente em banco de dados. Útil para testes que envolvem ordenação por data. Escrito em RUBY.
				\item ArmazenamentoDePacotes-Teste/server.rb\par
					Script que atua como uma espécie de servidor de arquivos, de forma extremamente simplificada, utilizando de socket TCP puro para a transferencia de arquivos. Escrito em RUBY.
				\item ArmazenamentoDePacotes-Teste/cliente.rb\par
					Script que atua como uma espécie de cliente para o servidor de arquivos, de forma extremamente simplificada, utilizando de socket TCP puro para a transferencia de arquivos. Utilizado para testar o "ArmazenamentoDePacotes-Teste/server.rb". Escrito em RUBY.
			\end{itemize}
				
    	
%    \begin{large}
%    \end{large} \par

%    \footnote{Isso aqui \'{e} pra estar no fim da p\'{a}gina.}
    \newpage
    \begin{large}
    	Desejo-lhe uma excelente experiência com o Visu-RA.
    \end{large}



\end{document}
